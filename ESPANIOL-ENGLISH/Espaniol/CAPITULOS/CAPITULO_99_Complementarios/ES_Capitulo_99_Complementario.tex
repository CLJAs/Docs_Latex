\chapter[Ordenados por referencias, no por aparición]{Comentarios Complementarios}

\section {Lo que pretendo que se deduzca:}

\noindent 1) Cuando todos los Packs que escogemos, de la relación, tienen cardinal uno, es un caso de equivalencia total con una función inyectiva.\\\\
\noindent 2) Los Packs pueden tener cardinales diferentes, y aún así respetarse las condiciones para que se cumpla el Teorema CA.\\\\
\noindent 3) Los Packs pueden tener cardinalidad transfinita.\\\\

En su día partimos de la sospecha que la diferencia entre $\mathbb{R}$ y $\mathbb{N}$ era los Irracionales. Que los números Irracionales formasen cadenas infinitas de símbolos era lo que provocaba que no se pudiesen encontrar biyecciones entre ambos conjuntos. Su diferencia no era realmente cardinal. Con eso en mente, y gracias a las estructuras que se pueden construir con las CLJAs, decidimos asignar un natural diferente a cada 'símbolo' del número Irracional. Si realmente eran únicos, alguna parte de su cadena debía ser única y diferente a la del resto de números Irracionales... y como había un natural por cada símbolo... en esa parte única tendríamos una serie de naturales únicos asignados a cada número Irracional.

De ahí que los Packs que usemos tengan cardinal transfinito. Cuando veamos la definición de la correspondencia flja\_abstracta, junto con la biyección $\Omega$ (La función flja de la Clja\_FTC), veremos como cada natural está asociado a cada 'etiqueta' de un SNEI (en cada universo). Incluso como la misma etiqueta, en la misma posición, recibe un natural diferente pq depende del resto de etiquetas anteriores. Sucederá que, ante la primera diferencia de etiquetas, aunque el resto sean idénticas en las mismas posiciones, la serie de números naturales asignados comenzará a ser diferente. A ser 'disjuntos'.

La responsable de esta propiedad será la CLJA, que tiene muchas similitudes con un grafo tipo árbol. Aunque dos nodos, tengan el mismo nombre, pero estén en diferentes ramas, en realidad serán nodos diferentes. Y una de las propiedades de la CLJA es poder asignar un natural único a cada uno de sus 'huecos azules' (el concepto equivalente a nodo).

Pero la distribución que genera es TAN caótica... que aparte de no tener sentido lexicográfico, ha creado la necesidad del concepto de Pack. Incluso siendo capaz de crear biyecciones a veces. O correspondencias que crean las condiciones para que podamos aplicar el Th CA. O ambas a la vez en un caos maravilloso. Entre los SNEFs y $LCF_{1}$, hay una biyección. Entre los SNEIs y $LCF_{2}$, hay una correspondencia... sobre la que estudiábamos si podía o no cmplir el Th CA. Pero todo parte de la misma CLJA.   

\newpage

\section {2: Segundo comentario}

Segundo comentario :D.

\newpage