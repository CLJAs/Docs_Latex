´\chapter{Introducción}

Vamos a intentar explicar a qué tipo de documento se enfrenta el lector.

Lo primero es aclarar que esto no es un paper. Por diversos motivos, como por ejemplo, no tener bibliografía. El motivo radica en que se me crea o no, muchas cosas son re-descubrimientos. Otras referencias me vienen por charlas informales con matemáticos a lo largo de varios años.

Pero en realidad el formato concreto no debería ser importante si el contenido es interesante.

Tampoco va a tener el formato teorema-demostración, aunque al menos si van a haber muchas definiciones. Cuando enunciemos una propiedad, trataremos de explicar por qué funciona. Ya he comprobado que esto tampoco es ningún problema, pues quién ha puesto un poco de interés, acaba juzgando muchas de ellas como obvias y triviales. Incluso hay gente que llega a afirmar que es incapaz de señalar dónde está el fallo de las afirmaciones que hago.

El motivo de esta última versión es que SI hay dos personas que han encontrado dos, posibles, fallos diferentes. La gracia está en que sus contra-argumentaciones son contradictorias. El argumento de cada uno, convierte en importante el punto que desconoce, y sirve para negar la contra-argumentación del otro. Y juntando ambos puntos se produce un fenómeno que, espero, sea considerado como extremadamente interesante.

También la experiencia me dicta que muchos van a tratar de insistir en que cite el fallo de las demostraciones de los teoremas que son el origen de este trabajo. Los ejemplos más simples son solo una pérdida de tiempo, y resumir el fenómeno numérico que realmente los pone en jaque, a mi, me resulta imposible. Nada obliga a que un "error" sea sencillo de explicar o puntualizar. Si lo fuese, ya se hubiese descubierto hace tiempo.

Este documento tiene la intención de presentar dicho fenómeno numérico, y que el lector juzque sus consecuencias. El fenómeno es real, es "mostrable" y es "explicable" pues todas sus propiedades, aunque sean muchas, son "triviales" y "obvias". Y no lo digo yo... son palabras de otra gente extraídas a regañadientes. Muchas de las evidencias usadas son innegables. Pero para entender el fenómeno, hay que entender muchos conceptos del contexto que le dio vida.

El contexto del fenómeno numérico es un intento de rebatir el Teorema de Cantor mediante un contra-ejemplo. Ese es su contexto, no el objetivo, por obligación del rigor, de este documento. Deberemos estudiar todas las herramientas diseñadas a tal efecto... para llegados a un punto, construir y explicar el fenómeno numérico.

Cómo un ultimo intento de mejorar la experiencia lectora, producto de críticas previas, durante el escrito veremos las marcas:\\\\
\noindent<comentario complementario: <<Numero::indicación de tipo>> > \\\\
\noindent Se supone que dicho comentario contendrá material NO NECESARIAMENTE RIGUROSO, ya que no debería ser tenido en cuenta a la hora de juzgar este trabajo. Ejemplos, anécdotas, experiencias personales sobre ciertas contra-argumentaciones solventadas, orígenes de los conceptos, etc... Para poder leer cada uno, bastará con acudir al capítulo que recopilará dichos comentarios, y buscarlo indicado por su referencia numérica, como un sub-apartado de dicho capítulo.

Así se podrá elegir si tener o no, una experiencia lo más aséptica posible. También lo más rigurosa posible, dadas mis capacidades.

Otro problema al que nos enfrentaremos es que no se trata de un trabajo lineal... es más bien un árbol de ideas. Procuraré ceñirme al hilo argumental de crear el fenómeno numérico, y clasificar el resto como comentarios complementarios. Pero uno de los problemas, que ya he visto en el pasado, es que al no mencionar otras ramas, se piense que lo construido es el límite de lo que se puede lograr, que no se comprenda bien el verdadero potencial de las construcciones LJA, o que se sienta la tentación de usar ciertas contra-argumentaciones. También de que la "ausencia" de una rama, cree la tentación de afirmar que no existe.

Así que esto pretende ser una recopilación de todos mis escritos, con un proceso nuevo añadido, después de darme cuenta de las contradicciones en ciertas contra-argumentaciones.

El capítulo de las Construcciones LJA, probablemente se divida en dos: lo estrictamente necesario para crear el fenómeno numérico, que de por sí es largo, y otro capítulo, probablemente también complementario, expandiendo la técnica y mencionando técnicas más avanzadas.

En caso de que se considere el fenómeno numérico como digno de estudio o de gran valía para la comprensión de las cardinalidades infinitas, hay muchas ramas abiertas, y soluciones a medias, que dependen precisamente de ese juicio. Pero intentaré alejarlas de la rama principal, pues son dependientes del juicio de la comunidad matemática, aunque yo tenga mi propia opinión al respecto de su valía. Y serán mencionadas con la esperanza de que sirvan como pistas, para futuros posibles estudios que cuenten con los recursos adecuados.

Y como se dice en el abstract, el fenómeno numérico es una diagonalización inversa. Una construcción de diagonalización entre dos conjuntos, con cardinalidades iguales a $\mathbb{N}$ y $P(\mathbb{N})$, donde intercambian sus papeles tradicionales, generando propiedades tremendamente similares a las diagonalizaciones conocidas.

Mi esperanza es que la existencia de dicho fenómeno lleve a dos posibles conclusiones a todo aquel que tenga la paciencia para estudiarlo:

1) El argumento original es erróneo: con él, se puede demostrar lo mismo y lo contrario.

2) O la existencia de esta construcción demuestra que tienen el mismo cardinal.

Pero ese juicio no quiero afirmarlo, primero observadlo, y luego decidid por vosotros mismos. Aunque yo pueda estar equivocado con las consecuencias, algunas herramientas que uso presentan ciertas innovaciones. Menos espectaculares que la conclusión que yo deseo que se comparta, pero interesantes en si mismas.

Advierto, en mi intención de ser honesto, que uno de los puntos más débiles será la "solución multiverso"... que en realidad tiene "sentido cardinal", pero retuerce demasiado las definiciones. Os corresponderá juzgar si las he retorcido demasiado, o entra dentro de parámetros aceptables. Esto y juzgar las similitudes con las propiedades de las construcciones de diagonalización. El resto es tan sólido, que aún enunciado sin rigor, mucha gente se ve obligada a admitir su validez.

Y si un fenómeno, o un ejemplo concreto, lleva a plantearnos la validez de teoremas, o incluso axiomas... es independiente de la validez de su existencia. Ésta debe ser juzgada de forma independiente, siempre y cuando no viole ninguno de ellos, sino que muestre sus posibles contradicciones. 


