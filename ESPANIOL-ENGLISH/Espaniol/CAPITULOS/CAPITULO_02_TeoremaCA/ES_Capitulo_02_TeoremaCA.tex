\chapter{Teorema CA}

\newpage
\section{Relaciones no aplicación}
Uno de las herramientas de este trabajo van a ser relaciones que no son aplicación: simples correspondencias. Y no serán 'función', porque por cada elemento del conjunto Dominio, a veces, tendremos relacionados más de un elemento del conjunto Imagen. Y no serán varios elementos dentro de un conjunto, sino que serán varios pares de la relación, cuyo elemento del Dominio es el mismo, pero cambian los elementos del conjunto Imagen, en cada par.

Por ejemplo:

\noindent Siendo el conjunto $X=\{a,b,c\}$ y el conjunto $Y=\{1,2,3,4,5,6,7,8,9\}$, y una relación $r:X \rightarrow Y$, los pares no serán del estilo:\\
$a \rightarrow \{1,2\}$, pues esto no sería una relación entre X e Y, sino entre X y un subconjunto de P(Y).\\\\
\noindent Sino que serán del estilo:\\
$a \rightarrow 1$\\
$a \rightarrow 2$\\\\
\noindent Formándose los pares (a,1) y (a,2)

No serán herramientas exclusivas, pues según el caso, usaremos inyecciones, biyecciones, o cualquier otra herramienta que nos permita obtener deducciones sobre los cardinales de los conjuntos envueltos. Siempre explicando el motivo por el cual nos permitimos la licencia de usar esa herramienta y no las conocidas. 

NO IMPORTA que las herramientas que usemos sean enrevesadas o poco elegantes, ni incluso que existan alternativas más sencillas... simplemente deben ser correctas y estar lo mejor definidas posible.

\newpage
\section{Concepto de Pack}
\noindent
Sea $f:X \rightarrow Y$\\ 
correspondencia entre conjuntos cualesquiera.\\

\noindent 
Sea $a \in X$ definimos:\\
$f(a) = \{ b \in Y / f(a) = b \}$\\ 
el conjunto imagen mediante la correspondencia $f$ del elemento $a$ de $X$.\\

\noindent 
Llamamos 'Packs' de $a$ a todo subconjunto NO VACÍO de $f(a)$.



\newpage
\section{Teorema CA}
Prueba de texto.


$\langle$Comentario Complementario: $\langle\langle$1:Lo que pretendo que se deduzca$\rangle\rangle$

\newpage
\section{Interpretación intuitiva}
Prueb

\newpage
\section{Equivalencias con el concepto de inyección}

El Teorema CA, ha acabado siendo una generalización no intencionada de la función inyectiva, en cuanto a su uso para poder afirmar que el cardinal de un conjunto, no es mayor que el cardinal de otro conjunto. Puesto que la inyectividad, es un caso particular del teorema.

Pero ahí se acaban las similitudes... sería un error estar buscando constantemente la alternativa inyectiva de las relaciones que usemos. Lo que buscaremos será abusar y retorcer las capacidades de las propiedades de los Packs con cardinal transfinito.

Hasta tal punto, que dónde matemáticamente fracasan los conceptos de inyectividad o biyectividad, los Packs transfinitos generarán fenómenos numéricos innegables. Fenómenos que espero generen muchas dudas como mínimo.