\newpage
\renewcommand{\abstractname}{Abstract}
\begin{abstract}
	
	%Testar la veracidad del Teorema de Cantor, no debería llevarnos a ningún lado excepto comprobar su solidez. Digamos que en el viaje realizado, buscando formas de ponerlo a prueba, me he encontrado una %serie de fenómenos numéricos.
	
	Testing the truthfulness of the Cantor's Theorem, should not drives us anywhere, except check up on its solidity. Let's say that in the travel done, searching ways to put it into the test, I have found a serie of numeric phenomenoms.
	
	%Algunos son solo redescubrimientos de cosas conocidas. Otros son cosas interesantes, aparentemente nuevas, pero sin demasiada relevancia en el campo de la teoría de conjuntos. Por ejemplo: haber %encontrado un patrón común entre diversas biyecciones famosas, creando alternativas al uso de números primos.
	
	Some of them are just rediscoverings of known things. Anothers, are very interisting, apparently new, but without too much relevance in the field of set theory. For example: having found a common pattern between several famous bijections, creating alternatives to the use of prime numbers 
	
	%Pero hay dos fenómenos numéricos bastante curiosos. Por separado, sobre cada uno de ellos, han opinado dos matemáticos diferentes sin conocer la existencia %del otro fenómeno. Lo curioso es que las contra-argumentaciones de ambos, se vuelven contradictorias, cuando mezclamos ambos fenómenos en uno: un intento %de diagonalización inversa. La contra-argumentación de uno, le quita la razón al otro y viceversa.
	
	But there are two quite curious numerical phenomena. Separately, about each of them, two different mathematicians have opined without knowing the existence of the other phenomenon. The curious thing is that the counter-arguments of both, become contradictory, when we mix both phenomena in one: an attempt of inverse diagonalization. The counter-argumentation of one, takes away the reason to the other and vice versa.
	
	%Un proceso, la diagonalización inversa, por el cual intentaremos 'afirmar' una consecuencia cardinal entre un conjunto, LCF, con el mismo cardinal que %$\mathbb{N}$ y otro conjunto, SNEIs, con el mismo cardinal que $P(\mathbb{N})$. La novedad es que invertiremos los papeles: partiremos de afirmar que SNEIs %tiene un cardinal mayor que LCF, y para que eso suceda, debe ser 'posible' hallar una 'muestra', muy concreta, de esa diferencia cardinal. La gracia va a %estar en que nos va a resultar totalmente imposible hallarla. Y para 'mostrarlo', dada la singularidad del caso, podremos usar argumentos tremendamente %similares a los de la diagonalización. 
	
	A process, reverse diagonalization, by which we will attempt to 'assert' a cardinal consequence between a set, LCF <List of Finite Paths::Lista de Caminos Finitos>, with the same cardinality as $\mathbb{N}$, and another set, SNEIs, with the same cardinality as $P(\mathbb{N})$. The novelty is that we will reverse the roles: We will start by affirming that SNEIs has a cardinal greater than LCF, and for that to happen, it must be 'possible' to find a very specific 'sample' of that cardinal difference. The grace is going to be that it is going to be totally impossible for us to find it. And to 'show' it, given the uniqueness of the case, we will be able to use wildly similar arguments to diagonalization.
	
	%Al final vamos a obtener un fenómeno con las mismas fortalezas y debilidades que dos técnicas de diagonalización usadas por Cantor. Y uso el término %'debilidades', pues las contra-argumentaciones a este fenómeno tienen 'traducción' directa en las diagonalizaciones cantorianas, donde no se consideran %'debilidades'. Las similitudes van a ser tremendamente asombrosas.
	
	In the end we are going to obtain a phenomenon with the same strengths and weaknesses as two diagonalization techniques used by Cantor. And I use the term 'weaknesses', since the counter-arguments to this phenomenon have a direct 'translation' in the cantorian diagonalizations, where they are not considered as 'weaknesses'. The similarities are going to be tremendously amazing.
	
	
	%Por eso, este documento consistirá en la exposición de varios fenómenos numéricos, y de explicar como funcionan en equipo para crear la susodicha diagonalización inversa.
	
	Therefore, this document will consist of exposing various numerical phenomena, and explaining how they work as a team to create the aforementioned inverse diagonalization.
	
	%A pesar de usar definiciones correctas, siempre comienzan diciéndome que son confusas, para acabar aceptando su validez. Simplemente por su densidad inicial. No siendo, las definiciones, totalmente formales, se entienden rápida y perfectamente. No hay forma de acelerar o suavizar el viaje, así que tendremos que ir con calma, explicando y definiendo los conceptos del contexto de dichos fenómenos. Va a ser triste, por la simpleza de muchos de ellos... pero la experiencia me indica que es un viaje inevitable. Juro que las mismas personas han pasado de decir cosas como 'ininteligible' o 'confuso', a cosas como 'obvio' o 'trivial', simplemente por no haberlo leído bien, dado el tema que 'sobrevuela' a mi trabajo.
	
	Despite using correct definitions, they always start by telling me that they are confusing, to end up accepting its validity. Simply because of its initial density. Not being, the definitions, totally formal, are quickly and perfectly understood. There is no way to speed up or smooth the trip, so we will have to go calmly, explaining and defining the concepts of the context of these phenomena. It will be sad, because of the simplicity of many of them ... but experience tells me that it is an inevitable journey. I swear that the same people have gone from saying things like 'unintelligible' or 'confusing', to things like 'obvious' or 'trivial', simply for not having read it well, given the topic that 'flies over' my work.
	
	%Una vez expuestos, que cada cual les encuentre el sentido que desee. Pero su existencia es innegable.
	
	Once exposed, that each one finds the meaning they want. But its existence is undeniable.
			
\end{abstract}